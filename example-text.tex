
\documentclass[10pt]{article}

\usepackage[frenchmath]{lorig}

%\renewcommand\Re{\operatorname{Re}}
%\renewcommand\Im{\operatorname{Im}}


%%%%%%%%%%%%%%%%%%%%%%%%%%%%%%%%%%%%%%%%%%%%%%%%%%%%%%%%%%%%%%%%%%%%%%%%%%%%%%%%%%%%
%
%          Weston's Macros
%
%%%%%%%%%%%%%%%%%%%%%%%%%%%%%%%%%%%%%%%%%%%%%%%%%%%%%%%%%%%%%%%%%%%%%%%%%%%%%%%%%%%%


% variables
%\newcommand{\xone}{X_t^{(1)}}
%\newcommand{\xtwo}{X_{t}^{(2)}}


\begin{document}
\section{Stochastic Volatility}

\subsection{Options and Implied Volatility}
Let us begin by introducing the following convenient notation:
\begin{align}
&\tau
	:= T - t,
&& x
	:= \log\ S_t,
& k
	:= \log\ K,
\end{align}
where $T,K$ are the fixed expiration date and strike price, respectively, of a European call or put option. In the 
Black-Scholes-Merton (BSM) setting, the price of an asset $S$ is a geometric Brownian motion  with dynamics given by
\begin{align}
\frac{d S_t}{S_t} 	
	= \mu\ dt + \sigma\ dW_t,
\end{align}
where $W_t$ is a Brownian motion under the physical measure $\Pb$. The parameters $\mu$ and $\sigma$ are both constant. 
The BSM pricing formula for a call option is well-known and given by 
\begin{align}
\label{eqn.BSMcallprice}
&u^{\text{BS}} (\sigma)
	:= e^x \mathcal{N} ( d_+ ( \sigma) ) - e^k \mathcal{N} ( d_- (\sigma)),
& d_\pm 
	:= \frac{1}{\sigma \sqrt{\tau}} \( x - k \pm \frac 12 \sigma^2 \tau \).
\end{align}

The BSM price \eqref{eqn.BSMcallprice} can be expressed as the expected payoff the option. However, this expectation is 
not taken with respect to the physical measure, but with respect to the \textit{risk-neutral measure} that makes 
$S_t$ a martingale i.e. the measure $\Qb$ under which
\begin{align}
\frac{d S_t}{S_t} = \sigma\ d W_t^{\Qb}.
\end{align}
The BSM price is thus
\begin{align}
u^{\text{BS}} (\sigma) 
	= \Eb^{\Qb} \[ \( S_T - K\)^+ \mid \log\ S_t = x \].
\end{align}

\begin{definition}[Implied Volatility]
Consider a given call option $u$ (which can be either an observed price or derived from a model). The \textit{implied 
volatility} of the call option is the unique positive solution $I$ of 
\begin{align}
u^{\text{BS}}(I) = u. 
\end{align}
\end{definition}

The map $I: \Rb\times \Rb \mapsto \Rb$ that takes $(T,k)$ to the implied volatility is called the implied volatility 
surface. If the market price of an option matched the BS prce, $I$ would be constant and equal to the stock's 
historical volatility $\sigma$. However, with equities data, the function $I(T,\cdot)$ exhibits a downward slope as $k$ 
increases. This is known as the \textit{implied volatility skew}. Generalizing the BS model can help to capture this 
behavior. 

\subsection{No-arbitrage Pricing and Risk-Neutral Measure}
In standard pricing theory, it is assumed that markets to not admit arbitrage. This implies that there exists a 
probability measure $\Qb$ under which all traded assets are martingales. This is typically referred to as the 
\textit{risk-neutral measure}. Consequently, the price $u_t$ of an option at time $t$ with payoff $\phi (S_T)$ at the 
time of expiration is give by 
\begin{align}
u_t
	= \Eb^{\Qb} \[ \phi( S_t) \mid \Fc_t \],
\end{align}
where $\Fc_t$ is the history of the market up to time $t$. There are typically nontraded sources of randomness. As a 
result, there exist infintely many risk-neutral measures. The nonuniquness of these measures is referred to as 
\textit{market incompleteness}, meaning not every derivative asset can be perfectly hedged. We usually assume that a 
market has chosen a specific risk-neutral measure consistent with observed option prices. In what follows, we will 
model asset dynamics under a risk-neutral pricing measure $\Qb$, which is assumed to be chosen by the market. Under 
$\Qb$, we have 
\begin{align}
& S_t 
	= e^{X_t}, 
&& dX_t 
	= - \frac 12 \sigma_t^2\ dt + \sigma_t dW_t^\Qb.
\end{align}

\subsection{Local Volatility Models}
A local volatility (LV) model assumes that the volatility, $\sigma_t$, of the underlying asset is a deterministic 
function $\sigma(t,\xi)$, evaluated at time $t$ and the time-$t$ value of the underlying asset $X_t$,
\begin{align}
\label{defn.LVmodel}
dX_t
	= - \frac 12 \sigma^2(t,X_t)\ dt
	  + \sigma( t, X_t)\ dW_t^\Qb.
\end{align}

One advantage of LV models is that markets remain \textit{complete}, meaning derivatives written on $S$ can be 
perfectly hedged (like the BS model). 

\begin{center}
\begin{tabular}{|l|l|l|l|}
\hline
 	 & $\sigma$ 				& Model & Completeness\\
\hline
Local Volatility & $\sigma(t,\xi)$     	& 
\shortstack{$ dX_t = -\frac 12 \sigma^2(t,X_t)dt + \sigma(t,X_t)dW_t^\Qb$} & Yes \\

\hline

Stochastic Volatility 	 & $\sigma(\upsilon)$ &  
\shortstack{$ dX_t = -\frac 12 \sigma^2(Y_t)dt + \sigma(Y_t)dW_t^\Qb$ \\
            $ dY_t =  \alpha(Y_t) dt + \beta (Y_t) d B_t^\Qb$\\
            $d\< W^\Qb, B^\Qb\> = \rho dt$}
& With continuous trading\\
\hline
Local-Stochastic Volatility		 & $\sigma(t,\xi,\upsilon)$ & 
\shortstack{$ dX_t = -\frac 12 \sigma^2(t,X_t, Y_t)dt + \sigma(t,X_t,Y_t)dW_t^\Qb$ \\
            $ dY_t =  \alpha(t,X_t,Y_t) dt + \beta (t,X_t,Y_t) d B_t^\Qb$\\
            $d\< W^\Qb, B^\Qb\> = \rho dt$}
& No? \\
\hline
\end{tabular}
\end{center}






\end{document}
